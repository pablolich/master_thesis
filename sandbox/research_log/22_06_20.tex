\documentclass[10pt,letterpaper]{article}
\usepackage[right=20mm,left=20mm,top=20mm,bottom=20mm]{geometry}
\usepackage{amsmath,amssymb}
\usepackage[utf8x]{inputenc}
\usepackage{graphicx}
\graphicspath{{../../results/}}
\usepackage{subcaption}
\usepackage{caption}
\usepackage{braket}

\begin{document}
	
    \LARGE{\textbf{Research log summary}}\hfill\Large{June 22, 2020}
    \section*{Aim}
    	I would like to adress one of these two questions (or both if I have time):\\
    	\begin{itemize}
    		\item Precict the outcome (what community will dominate, and in what proportion) of the coalescence of two communities when mutualisms are considered.
    		\item Will members of communities with a history of coalescence have a higher persistence up on interaction with a 'naive' community?\\
    	\end{itemize}
    	
    \section*{Progress}
		\begin{itemize}
			\item Implemented simulation of a coalescence event.
			\item Calculated more candidates to predict coalescence output (initial growth rate, F)
			\item Simulated 1000 communities and performed coalescence events of all possible pairs of the same richness (avoid selection effects)
			\item When C1 coalesces with C2, the community outcome is C. The similarity of C1 and C was calculated as the normalized scallar product of their species abundance vector.
			\begin{equation}
				S(C_1, C) = \frac{\vec{s_1} \cdot \vec{s}}{\sqrt{|\vec{s_1}|}\sqrt{|\vec{s}|}}
			\end{equation}
			Plotting fitness difference $ \Delta F = F  -  F_1 $ versus the similarity, shows that F is not a good predictor of the coalescence outcome in communities where mutualisms exist.
			\begin{figure}[h]
				\centering
				\includegraphics[width=0.6\linewidth]{similarity_fitness.pdf}
				\caption{Fitness difference versus similarity between each pair of competing communities.}
				\label{fitness_similarity}
			\end{figure}
			\item Histogramed similarity and found that 0 and 1 are the most common cases. This is what I expected, since including mutualisms makes the community a more cohesive unit, so that when it competes with another one, most of the times the outcome is one of the pure communities, and a perfect mixing ($ S = 0.5 $) is the least likely option.
			\begin{figure}[h]
				\centering
				\includegraphics[width = 0.5\linewidth]{similarity_hist.pdf}
				\caption{Similarity histogram for each coalescence event}
			\end{figure}
			
    		\end{itemize}
    	\section*{Next steps}
		    \begin{itemize}
		    	\item Record more candidates for predictor of community coalescence outcome: Total flux (un-weighted), community cohesion, and flux per metabolite.
		    	\item Check for correlations between similarity and each of the candidates.
		    \end{itemize}
    \section*{Questions}
    	\begin{itemize}
    		\item Any other predictors you can think of?
    		\item Is the community cohesion measure a promising option?
    	\end{itemize}


	
		
		
\end{document}