\documentclass[10pt,letterpaper]{article}
\usepackage[right=20mm,left=20mm,top=20mm,bottom=20mm]{geometry}
\usepackage{amsmath,amssymb}
\usepackage[utf8x]{inputenc}
\usepackage{graphicx}
\graphicspath{{../../results/}}
\usepackage{subcaption}
\usepackage{caption}
\usepackage{braket}

\begin{document}
	
    \LARGE{\textbf{Research log summary}}\hfill\Large{June 22, 2020}

    \section*{Progress}
		\begin{itemize}
			\item Calculated facilitation matrix and competition matrix as follows:  \\
			For the facilitation matrix, $ F $ the facilitation index $ f_{ij} $ that species $ i $ has with species $ j $ is measured by the normalized cardinality of the intersection of the set of products produced by species $ i $, $ P_i $ with the set of substrates consumed by species $ j $, $ S_i $. The  normalizing factor is the maximum number of metabolites that they can share.\\
			\begin{equation}
				f_{ij} = \frac{|P_i \cap S_j|}{\min (|S_i|, |P_j|)}
			\end{equation}
			For the competition matrix $ C $, the competition index $ c_{ij} $ between $ i $ and $ j $is measured by\\
			\begin{equation}
				c_{ij} =  \frac{|S_i \cap S_j|}{\min (|S_i|, |S_j|)}
			\end{equation}
			Note that $ F $ is not symetric, but $ C $ is. 
			\item For each species, I calculated the give ($ g $), receive ($ r $)  and competition ($ w $) indexes as $ g_i = \sum_{j} f_{ij}  $,  $  r_j = \sum_{i}f_{ij} $, and $ w_j = \sum_i c_{ij}$. 
			\item Plotted the abundance at stable state of each species, against their respective indexes for regimes where the number of reactions can be high ($ n_{reac} \sim Uniform(1, {m \choose 2}) $) or low  ($ n_{reac} \sim Uniform(1, m) $) 
			\item For each community, I also plotted the generosity index $ r - g $ for each species, against its abundance at equilibrium. I found that generous species tend to do better.
			\item I plotted the median abundance accross all simulations against the number of reactions of each strain. I found that number reactions is a proxy for abundance at equilibrium, since less reactions yield higher abundances.
			\item I tested the predictor of community coalescence $ P = \frac{\braket{r - g}}{\braket{c}}$ to see whether in a coalescence event, cohesion (generosity) is maximized, and competition is minimized, ie, communities with higher P do better in a community coalescence event. I found nothing.
			\begin{figure}[h]
				\centering
				\includegraphics[width=0.6\linewidth]{providing_high_regime.pdf}
				\caption{Abundance versus $ g $ inidex for high reaction number regime}
			\end{figure}
			\begin{figure}[h]
				\centering
				\includegraphics[width = 0.5\linewidth]{facilitation_high_regime.pdf}
				\caption{Abundance versus $ r $ index for high reaction number regime.}
			\end{figure}
				\begin{figure}[h]
				\centering
				\includegraphics[width = 0.5\linewidth]{competition_high_reac_regime.pdf}
				\caption{Abundance versus $ c $ index for high reaction number regime.}
			\end{figure}
		
		
		
			\begin{figure}[h]
			\centering
			\includegraphics[width=0.6\linewidth]{providing_low_regime.pdf}
			\caption{Abundance versus $ g $ inidex for high reaction number regime}
			\end{figure}
			\begin{figure}[h]
				\centering
				\includegraphics[width = 0.5\linewidth]{facilitation_low_regime.pdf}
				\caption{Abundance versus $ r $ index for high reaction number regime.}
			\end{figure}
			\begin{figure}[h]
				\centering
				\includegraphics[width = 0.5\linewidth]{competition_low_regime.pdf}
				\caption{Abundance versus $ c $ index for high reaction number regime.}
			\end{figure}
			\begin{figure}[h]
				\centering
				\includegraphics[width = 0.5\linewidth]{reaction_fitness.pdf}
				\caption{Abundance versus $ c $ index for high reaction number regime.}
			\end{figure}	
    		\end{itemize}
    	\section*{Next steps}
    	
		    \begin{itemize}
				\item Plot time series for community coalescence.                          
				\item Check what are things like in the eregime of high number of reactions.
				\item To check for higher interactions with distant neighbours, represent the facilitation matrix as a network and visualy inspect it
				\item See if the total flux through of the community drives community coalescence.
				\item Record evolution of competititon, facilitation and providing through time, to see deeply what changes in the dynamics.
				\item What resource flux goes through each metabolite at each timepiont?   
				\item Histogram the degree distribution of these networks                  
				\item Try the diversification option                                       
				\item Plot heatmap with number of reactions, and where are they in the metabolite space
		    \end{itemize}

	
	
		
		
\end{document}