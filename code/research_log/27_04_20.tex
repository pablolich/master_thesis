\documentclass[10pt,letterpaper]{article}
\usepackage[right=20mm,left=20mm,top=20mm,bottom=20mm]{geometry}
\usepackage{amsmath,amssymb}
\usepackage[utf8x]{inputenc}
\usepackage{graphicx}
\graphicspath{{../../results/}}
\usepackage{subcaption}

\begin{document}
	
    \LARGE{\textbf{Research log summary}}\hfill\Large{Apr 27,2020}
    \section*{Progress}
	\begin{itemize}
	    \item Implemented first version of numerical integration of the model with the following parameters:\\
	    \begin{center}
	    	\begin{tabular}{ |c|l|c| } 
	    		\hline
	    		 Parameter & Meaning & Value \\
	    		\hline
	    		m &Number of metabolites &  100 \\ 
	    		s &Number of strains &  10 \\ 
	    		$ \Delta G_{ATP} $ &ATP Gibbs energy & $ 7.5\cdot 10^{4} $\\ 
	    		$ \mu_0 $ & Most energetic metabolite &  $ 3\cdot 10^{4} $ \\ 
	    		nATP &max$\left(\dfrac{\Delta G^{0}_{S\rightarrow P}}{\Delta G_{ATP}}\right)$ &  4 \\ 
	    		$ \eta $ &Moles of ATP energy per reaction &  0.5  \\ 
	    		$ q_{max} $ &Maximum reaction rate &  1 \\ 
	    		$ K_s $ &Saturation constant &   0.1 \\ 
	    		$ k_r $ &Reversibility constant &  10 \\ 
	    		g & Growth factor & 1 \\ 
	    		m  & Maintenance factor &   0.2$\cdot J_{grow}$ \\ 
	    		$ \kappa $  & Externally supplied resource &  1 \\ 
	    		$ \gamma $ &Dilution rate &   0.5 \\ 
	    		$ N_{0}$ & Populations initial conditions &   (1, 1, ..., 1)\\ 
	    		$ C_{0}$ & Concentrations initial condition &   (0, 0, ..., 0)\\
	    		\hline
	    	\end{tabular}
	    \end{center}
    
    	The simulation with this parameters yields the time-series population and metabolite concentration dynamics seen in figure \ref{fig:dynamics}.
    	\begin{figure}[h]
    		\centering
    		\includegraphics[width=0.7\linewidth]{dynamics_example.pdf}
    		\caption{Time series of population and concentration dynamics}
    		\label{fig:dynamics}
    	\end{figure}
    	If I run my simulation with the most energetic metabolite being $ \mu_0 =  3\cdot 10^{6}$, which is biologically more realistic, I my populations and concentrations diverge. I am currently investigating why
    	\item Checked reaction networks  for different values of nATP. Results shown in figure \ref{fig:network}
    	\begin{figure}
    		\centering
    		\includegraphics[width=\linewidth]{reac_matrix_tot.pdf}
    		\caption{Mean of eaction network matrix facross strains for $ n_{ATP} = 4 $ (top) , $ n_{ATP} = 16 $ (left), $ n_{ATP} = 32 $ (right). Note that the closer the elements to the diagonal, the lower the gibbs energy change per reaction is. }
    		\label{fig:network}
    	\end{figure}
	\end{itemize}
    \section*{Questions}
	\begin{itemize}
	    \item What are the details of the generation of a set of random chemical potentials?
		My random chemical potentials, generate Keq that vary from 10$^{-7}$ up to 10$^{45}$. These are not biologically realistic values of equilibrium constants. I think the problem is that my chemical potential intervals vary a lot, leading to very high or very low $\Delta G^{0}_{S\rightarrow P}$ and therefore affecting $K_{eq}$ through\\
		\begin{equation}
			K_{eq}(S, P) = \exp\left(\frac{-\Delta G^{0}_{S\rightarrow P} - \eta\Delta G_{ATP}}{RT}\right)
		\end{equation}
		\item I am finishing reaction networks when the last metabolite is reached, and I don't know if this is a realistic constraint.
	\end{itemize}

	

		\begin{figure}[h]
		    \centering
		    \includegraphics[width = \textwidth]{question.pdf}
		    \caption{Histograms of $ \Delta G^{0}_{S\rightarrow P} $ and $ \ln(K_{eq}) $. Note that $ \Delta G^{0}_{S\rightarrow P} $ is bounded between 0 an $  4\Delta G_{ATP} $, and still high $ K_{eq} $ are obtained. }
		\end{figure}
	
	\section*{Next steps}
		\begin{itemize}
			\item Investigate why my dynamics are diverging.
			\item Understand Jacob's way of generating reaction networks and $ \eta $ and $ K_{eq} $ bounds
			\item 	Change the way I generate reaction networks to make it quicker.
		\end{itemize}
	
		
		
\end{document}
