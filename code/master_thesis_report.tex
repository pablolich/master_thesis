\documentclass[titlepage,11pt]{article}
\usepackage{graphicx} %To include figures
\graphicspath{{../Results/}}
\usepackage{lineno} % To count lines
\usepackage{setspace} %To change line spacing
\usepackage{cite} % To cite
\usepackage{amsmath}
\usepackage[paper=a4paper,margin=1in]{geometry}

\newcommand{\wordcount}{\input{count2.sum}} %Word counting

\doublespacing


\begin{document}
	
	\title{\textbf{Master thesis title} }
	\author{Pablo Lechón Alonso \\ [30pt]
		Imperial College London}
	\date{Word count: \wordcount}%We don't want to show the date
	\maketitle
	
	\newgeometry{top=1.5in,bottom=1.5in,right=1.5in,left=1.5in} %Change geometry
	
	\begin{abstract}
	
	\end{abstract}
	
	\tableofcontents
	\newpage
	
	\begin{linenumbers}
		\section{Introduction}
			\begin{itemize}
				\item State my aims/hypotheses/questions by the end of the introduction.
				\item Make sure I explain everything adequately. Provide more background in the introduction and methods that
				\item What is the problem I am tackling? you end up looking forward to read more
				\item the general way the problem has been approached.
				\item Clearly define my aims of the rearch project.
				\item Why is it interesting? Why don't we know the answer?
				\item Build from the most general and fundamental hypotheses to the most refined or teneous ones.
				\item How will I go about testing my hypothesis.
				
			\end{itemize}
			Following, what I want to talk about
			\begin{itemize}
				\item Microbial Community Coalescence in communities with mutualisms ie, cross-feeding.
				\item Thermodynamic constraints
				\item What drives a community to be succesful in a coalescence event?
				\item Will a coalesced community be more perssistent than a naive one upon a  event if it has a history of coalescence?
				\item How cohesive a community is when we analyze it in terms of cohord and dominant species? 
				\item Experimental studies report cohesivness in comunity coalescence events.
			\end{itemize}
	\end{linenumbers}
		
		\newpage
		\bibliographystyle{unsrt}
		\bibliography{/Users/pablolechon/library}
	\end{document}